
\chapter{Position of the Project}
\label{chap:project}


\section{Resume}

\section{Technical choice}

\section{Forecasting organization}

\subsection{Kanban}

To realize this project, I decided to use some tools to arrange my work. First
of all, I decided to use a agile technique of management which name Kanban.~\\

\definition{Kanban}{Kanban is a new technique for managing a software
  development process in a highly efficient way. Kanban underpins Toyota's
  "just-in-time" (JIT) production system. kanban system consists of a big board
  on the wall with cards or sticky notes placed in columns with numbers at the
  top \cite{peterson:kanban}}


Kanban is an inventory-control system to control the supply chain. It use a
board with columns. Each column represent a status, for example: to do, doing,
done. In each column we put <<notes>> which represent a task. Moreover, each
column have a maximum number of notes authorized.






%%% Local Variables:
%%% mode: latex
%%% TeX-master: "../rapport_de_base"
%%% End:
