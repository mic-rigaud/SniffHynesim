
\chapter{Position of the Project}
\label{chap:project}


\section{Resume}

This project is to automate a sniffer to detect attack. The most important fact is that our sensors need to detect
a strategy of attack and not only patterns of attack. After our bibliographic studies, it is simple to understand
that we will use an IDS with an anomaly detection method. We will use SELKS\footnote{A description of SELKS is
  available page \pageref{chap:selks}} because it is the main free open source
IDS available on the market. A description of our technical choice is explain on the next section.

As it is suggest in the description of the project deliver by our professor, we also use Hynesim to automate test
and simulate our solution. In this way, we will have the ability to justify the effectiveness of our solution.
To do that we will run many possible attack scenarios.

\section{Technical choice}

We will firstly realize a network infrastructure on Hynesim. We will do a basic infrastructure with only one server.
Then, we will put on our network a SLEKS server as a simple IDS and we will configure. We choose, to implement firstly
a pattern matching method. We will also put an attacker on the network and we will realize simple attack to test our
infrastructure. It is possible to see the infrastructure on the figure \ref{fig:network_hynesim}.

\begin{figure}[h]
  \centering
  \includegraphics[width=0.7\textwidth]{reseau_hynesim}
  \caption{Hynesim network architecture}
  \label{fig:network_hynesim}
\end{figure}

After this installation, we will try to improve the system. we will implement an anomaly detection method. In this
way, the IDS will have the ability to detect a will to attack. One of the most important difficulty will be to not
raise alert for a normal utilization.

\section{Forecasting organization}

\subsection{Kanban}

To realize this project, we decided to use some tools to arrange our work. First of all, we decided to use a agile
technique of management which name Kanban.~\\

\definition{Kanban}{Kanban is a new technique for managing a software development process in a highly efficient
  way. Kanban underpins Toyota's "just-in-time" (JIT) production system. kanban system consists of a big board on
  the wall with cards or sticky notes placed in columns with numbers at the top \cite{peterson:kanban}}


Kanban is an inventory-control system to control the supply chain. It use a board with columns. Each columns
represent a status, for example: to do, doing, done. In each column we put <<notes>> which represent a task.
Moreover, each column have a maximum number of notes authorized.


\begin{figure}[h]
  \centering
  \includegraphics[width=\textwidth]{kandboard}
  \caption{Kanboard of this project}
  \label{fig:kanboard}
\end{figure}


Limiting the amount of task, at each step in the process, prevents overproduction and revels bottlenecks
dynamically. In fact, with this technique it is possible to have a better overview of the project and control it
dynamically.
~\\

For this project, we use Kanboard\cite{guillot:kanboard} self-hosted on our own Yunohost server.\footnote{ It is possible to
  see our kanboard at this link:
  \url{https://mic-rigaud.fr/kanboard/?controller=BoardViewController&action=readonly&token=10ea65eca908023dbcd8bc8dce75791c7a14d67912627dafaa5b71033222}}.
You can see at the figure \ref{fig:kanboard} the kanboard of this project. Each color represent a category of task:
blue: improvement , purple: installation, red : experience, green: status report, and gray: report.

With this tool, it is also possible to see tasks as a Grantt diagram. The figure \ref{fig:grantt} present our
Grantt diagram.

\begin{figure}[h]
  \centering
  \includegraphics[width=\textwidth]{grant}
  \caption{Grantt diagram of the project}
  \label{fig:grantt}
\end{figure}


\subsection{Github}

To realize this project, we also decided to use Git and Github as Git server. ~\\

\definition{Git}{Git is a version control system for tracking changes in computer files and coordinating work on
  those files among multiple people.}

Git permit to control version of our work and have a real showcase of it for our tutor. The figure
\ref{fig:github} is a screenshot of our Github server.

\begin{figure}[h]
  \centering
  \includegraphics[width=\textwidth]{github}
  \caption{Github server}
  \label{fig:github}
\end{figure}




%%% Local Variables:
%%% mode: latex
%%% TeX-master: "../rapport_de_base"
%%% End:
