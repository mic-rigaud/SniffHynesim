\chapter{IDS}

In this subject, we will have to create probe to detect attack. It is the aim of IDS that we will present.

\section{Presentation}

\definition{IDS}{An intrusion detection system (IDS) inspects all inbound and outbound network activity and
identifies suspicious patterns that may indicate a network or system attack from someone attempting to break into
or compromise a system. \cite{webopedia}}



There is many type of IDS:
\begin{itemize}
\item NIDS, network IDS. They listen and analyze the network and detect attack from network packet. They are the
  most interesting for our subject, so this document will focus primarily on this type of IDS.
\item HIDS, host IDS. These IDS are on a system and they detect intrusion inside it.
\item Hybrid IDS. They are composed with NIDS and HIDS.
\item IPS. They are NIDS with active functions which permit to stop attackers.
\item KIDS/KIPS, kernel IDS. They are type of HIDS. They are on the system kernel. They are more effective and
slower than HIDS.
\end{itemize}

In the following of this document we will talk about NIDS. But to detect attack there is also many methods.

%%%%%%%%%%%%%%%%%%%%%%%%%%%%%%%%%%%%%%%%%%%%%%%%%%%%%%%%%%%%%%%%%%%%%%%%%%%%%%
% TODO: mettre en avant les but d un ids. ie rapidite et pas de fausse alarm
%%%%%%%%%%%%%%%%%%%%%%%%%%%%%%%%%%%%%%%%%%%%%%%%%%%%%%%%%%%%%%%%%%%%%%%%%%%%%%

\section{Detection methods}

\subsection{Misuse detection}

This technique is the simplest. It use attack signature to raise alert. In fact, all attacks have a particularity,
if we detect this particularity we can detect the attackers. There is three sub methods.

\subsubsection{Pattern matching}

In this technique we have a based of signature and the IDS is looking for the pattern. If the pattern match
perfectly, this IDS raise an alert.

The problem is, only attacks which are in our based can be detected. So if there is a new attack (zero day), or if
the attack is not perfectly the same, we can't detect it.

However, this method is much used because it is high-performance, and with this method the IDS don't raise a lot of
false alerts

\subsubsection{Dynamic pattern matching}

In this techniques the IDS is also based on signature but this data base is dynamic. In other words, the IDS has
the faculty of adaptation and learning. The IDS improve his data base of signature automatically.

\subsubsection{Protocol analysis}

The last sub method we will present is the protocol analysis. This technique is based on the verification of
protocol. The IDS will check if flows are compliant with RFC\footnote{Requests for Comments, is a type of
  publication from the Internet Engineering Task Force (IETF) and the Internet Society (ISOC), the principal
  technical development and standards-setting bodies for the Internet.} standards. It will verify parameters of
packets and fields of them. An IDS can check many protocol as FTP, HTTP, ICMP, \dots

The advantages of this methods is that we can detect unknown attacks in contrary of pattern matching. However,
software publishers don't often respect RFC so we this technique is not always very effective.





\subsection{Anomaly detection}

This technique consists in detecting an intrusion with the analysis of the user's past behavior. So the IDS should
create a profile of users from his use and raise alert when there is an event outside this profile. To create
profile the IDS could use machine learning.
~\\

\begin{tabular}{|p{0.45\textwidth}|p{0.45\textwidth}|} \hline
Advantages                                                                                                 & Drawbacks \\ \hline The IDS should be able to detect every type of
attack even unknown (zero days) attacks.                                                                   & This method is not reliable. Every
alteration of the use create an alert                                                                                  \\
  \hline The IDS is autonomous                                                                             & This method
need a learning period. In fact, this method need to learn the habit of users,
so we need a period without attacks                                                                                    \\ \hline & An hackers can need only time. In fact,
if the attacker arrive to create a new profile after month with his attacks, he
could attack silently                                                                                                  \\ \hline

\end{tabular}


\subsubsection{Probabilistic method}
Bayésien network is a learning machine based on probability. The IDS will create a probabilistic model and if the
user and will raise an alert if the user don't respect this model.

For example, we know that in 90\% of cases in an HTTP request the first parameter is GET after a connection to the
port 80.


%
% \subsubsection{Statistic method}
% \subsubsection{Other method}

% It is possible to use every type of machine learning. In particularly, neural
% network.


\section{Main IDS}

There is many IDS on the market. The most popular open source solution are Snort, Suricata, Bro, Fail2Ban, ACARM
... There is also some tools <<all in one>>. It is usually OS\footnote{Operating system} with an IDS, a tool to
analyze alert, a tool to create rules,... The most popular are SELKS and ELKS. A description of SELKS is available
page \pageref{chap:selks}.




%%% Local Variables:
%%% mode: latex
%%% TeX-master: "../rapport_de_base"
%%% End:
