\chapter{IDS}

In this subject, a probe will be created to detect attacks. It is the aim of IDS that I are going to introduce.

\section{Presentation}

\definition{IDS}{An intrusion detection system (IDS) inspects all inbound and outbound network activity and
identifies suspicious patterns that may indicate a network or system attack from someone attempting to break into
or compromise a system. \cite{webopedia}}



There are many type of IDS:
\begin{itemize}[itemsep=0pt]
\item NIDS, network IDS. They listen and analyze the network and detect attacks from network packets. They are the
  most interesting for our subject, so this document will focus primarily on this type of IDS.
\item HIDS, host IDS. These IDS are on a system and they detect intrusion within it.
\item Hybrid IDS. They are composed of NIDS and HIDS.
\item IPS. They are NIDS with active functions which help to stop attackers.
\item KIDS/KIPS, kernel IDS. They are types of HIDS. They are on the system kernel. They are more effective and
slower than HIDS.
\end{itemize}

In the following document I will talk about NIDS. But to detect attacks there is also many methods.
~\\


To be efficient IDS should have a good balance between some features.
\begin{itemize}[itemsep=0pt]
\item \textbf{Speed:} In fact an IDS should analyze packets as fast as possible, otherwise, it will behind the network
  traffic.
\item \textbf{False alarm:} An IDS raises an alert when it detect attacks. But it could raise alarm during a normal
  utilization, a false alarm. It is one of the most important features, because at every alarm a system
  administrator needs to analyze the alert. So in company, every alarm cost time and money.
\item \textbf{Probability of detection:} For an IDS, the capacity of detecting attacks. The higher the probability
  the more the IDS will raise false alarms. In some sensitive systems, I prefer to detect every attack and raise
  many false alarms. That depends on the system.
\end{itemize}


\section{Detection methods}

\subsection{Misuse detection}

This technique is the simplest. It uses attack signature to raise alerts. In fact, all attacks have a particularity,
if I detect this particularity I can detect the attackers. There are three sub methods.

\subsubsection{Pattern matching}

In this technique I have a base of signature and the IDS looks for the pattern. If the pattern matches perfectly,
this IDS raises an alert.

The problem is, only attacks which are in our base can be detected. So if there is a new attack (zero day), or if
the attack is not perfectly the same, it cannot be detected.

However, this method is much used because it is high-performance, and with this method the IDS don't raise a lot of
false alerts

\subsubsection{Dynamic pattern matching}

In this techniques the IDS is also based on signature but this data base is dynamic. In other words, the IDS has
the faculty of adaptation and learning. The IDS improve its data base of signature automatically.

\subsubsection{Protocol analysis}

The last sub method I will present is the protocol analysis. This technique is based on the verification of
protocol. The IDS will check if flows are compliant with RFC\footnote{Requests for Comments, is a type of
  publication from the Internet Engineering Task Force (IETF) and the Internet Society (ISOC), the principal
  technical development and standards-setting bodies for the Internet.} standards. It will verify parameters of
packets and fields. An IDS can check many protocols such as FTP, HTTP, ICMP, \dots

The advantages of this method is that I can detect unknown attacks contrary to pattern matching. However,
software publishers don't often respect RFC so this technique is not always very efficient.





\subsection{Anomaly detection}

This technique consists in detecting an intrusion through the analysis of the user's past behavior. So the IDS should
create a profile of users from his use and raises alerts when there is an event outside this profile. To create a
profile the IDS could use machine learning.
~\\

\begin{tabular}{|p{0.45\textwidth}|p{0.45\textwidth}|} \hline
Advantages                                                                                                 & Drawbacks \\ \hline The IDS should be able to detect every type of
attack even unknown (zero days) attacks.                                                                   & This method is not reliable. Every
alteration of the use creates an alert                                                                                  \\
  \hline The IDS is autonomous                                                                             & This method
needs a learning period. In fact, this method needs to learn the habit of users,
so I need a period without attacks                                                                                    \\ \hline & Hackers can need only time. In fact,
if the attacker creates a new profile after many month with his attacks, he
could attack silently                                                                                                  \\ \hline

\end{tabular}


\subsubsection{Probabilistic method}
Bayésien network is a learning machine based on probability. The IDS will create a probabilistic model and will
raise an alert if the user don't respect this model.

For example, I know that in 90\% of cases in an HTTP request the first parameter is GET after a connection to the
port 80.



\section{Main IDS}

There are many IDS on the market. The most popular open source solutions are Snort, Suricata, Bro, Fail2Ban, ACARM
... There are also some tools <<all in one>>. It is usually OS\footnote{Operating system} with an IDS, a tool to
analyze alert, a tool to create rules,... The most popular are SELKS and ELKS. A description of SELKS is available
page \pageref{chap:selks}.




%%% Local Variables:
%%% mode: latex
%%% TeX-master: "../rapport_de_base"
%%% End:
