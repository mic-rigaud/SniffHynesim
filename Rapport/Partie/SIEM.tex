
\chapter{SIEM}
\label{chap:project}

After some new discussion with my tutor, we decide to had a new component in our architecture. So we will add a
SIEM to analyze all alert and take care of server's log. So we will add in this status report a part about SIEM
which was not initially.

\section{Description}

\definition{SIEM}{In the field of computer security, security information and event management (SIEM) software
  products and services combine security information management (SIM) and security event management (SEM). They
  provide real-time analysis of security alerts generated by network hardware and
  applications.\cite{wikipedia17:_secur}}


So SIEM are software which aggregate all security information, analyze them, and display them for user.

\section{Capabilities}

\begin{itemize}
\item \textbf{Data aggregation} aggregate data from many sources: network, IDS, log, applications, etc\dots
\item \textbf{Correlation} Analyze alerts and events to correlate them to spot an attack.
\item \textbf{Reporting and alerting} Create alert and reporting more clear.
\item \textbf{Visualization} SIEM permit visualization of all event with for example dashboard.
\item \textbf{Log management} Store all log in a central location. This permit to do forensic many month after
  events.
\end{itemize}


\section{Main SIEM}

There is many SIEM on the market. The most popular open source solutions are Prelude, OSSIM, and Cyberoram.






%%% Local Variables:
%%% mode: latex
%%% TeX-master: "../rapport_de_base"
%%% End:
