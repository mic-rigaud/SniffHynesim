\documentclass[a4paper, 11pt, oneside, oldfontcommands]{memoir}

%%%%% Packages %%%%%
\usepackage{lmodern}
\usepackage{palatino}
\usepackage[T1]{fontenc}
\usepackage[utf8]{inputenc}
\usepackage[english]{babel}


%%%%%%%%%%%%%%%%%%%%  PACKAGE SECONDAIRE

%\usepackage{amstext,amsmath,amssymb,amsfonts} % package math
%\usepackage{multirow,colortbl}	% to use multirow and ?
%\usepackage{xspace,varioref}
\usepackage[linktoc=all, hidelinks]{hyperref}			% permet d'utiliser les liens hyper textes
\usepackage{float}				% permet d ajouter d autre fonction au floatant
%\usepackage{wrapfig}			% permet d avoir des image avec texte coulant a cote
%\usepackage{fancyhdr}			% permet d inserer des choses en haut et en bas de chaque page
\usepackage{microtype}			% permet d ameliorer l apparence du texte
%\usepackage[explicit]{titlesec}	% permet de modifier les titres
\usepackage{graphicx}			% permet d utiliser les graphiques
\graphicspath{{./images/}}		% to say where are image
%\usepackage{eso-pic} 			% to put figure in the background
\usepackage[svgnames]{xcolor}	% permet d avoir plus de 300 couleur predefini
%\usepackage{array}				% permet d ajouter des option dans les tableaux
%\usepackage{listings}			% permet d ajouter des ligne de code
%\usepackage{tikz}				% to draw figure
%\usepackage{appendix}			% permet de faire les index
%\usepackage{makeidx}			% permet de creer les index
%\usepackage{fancyvrb}			% to use Verbatim
%\usepackage{framed}				% permet de faire des environnement cadre
%\usepackage{fancybox}			% permet de realiser les cadres
\usepackage{titletoc}			% permet de modifier les titres
%\usepackage{caption}
\usepackage[a4paper, top=2cm, bottom=2cm]{geometry}
\usepackage{frbib}                      %permet d avoir une biblio francaise
\usepackage[babel=true]{csquotes}
\usepackage{enumitem}

\usepackage{graphicx}
\RequirePackage{pageGardeEnsta}	% permet d avoir la page de garde ensta

\setcounter{secnumdepth}{2}		% permet d'augmenter la numerotation
%\setcounter{tocdepth}{2}		% permet d'augmenter la numerotation

%%%%%%%%%%%%%%%%%%  DEFINITION DES BOITES
\newcounter{rem}[chapter]

\newcommand{\remarque}[1]{\stepcounter{rem}\noindent\fcolorbox{OliveDrab}{white}{\parbox{\textwidth}{\textcolor{OliveDrab}{
\textbf{Remarque~\thechapter.\therem~:}}\\#1}}}

\newcounter{th}[chapter]

\newcommand{\theoreme}[2]{\noindent\fcolorbox{FireBrick}{white}{\stepcounter{th}
\parbox{\textwidth}{\textbf{\textcolor{FireBrick}{Théorème~\thechapter.\theth~:}}{\hfill \textit{#1}}\\#2}}}

\newcommand{\attention}[1]{\noindent\fcolorbox{white}{white}{\parbox{\textwidth}{\textcolor{FireBrick}{
        \textbf{Attention !}}\\\textit{#1}\\}}}


\newcommand{\definition}[2]{\noindent\fcolorbox{OliveDrab}{white}{\stepcounter{th}
    \parbox{\textwidth}{\textbf{\textcolor{OliveDrab}{Définition~\thechapter.\theth~:}}{~
        \textit{#1}}~\setlength{\parindent}{15pt}\par#2\par}}\par~\\}


%%%%%%%%%%%%%%%%%%%%%%%%%%%%%%%%%%%%%%%%%%%%%%%%%%%%%%%%%%%%%%%%%%%%%%%%%


%% INDEX %%%%%%%%%%%%%%%%%%%%%%%%%%%%%%%%%%%%%%%%%%%%%%%%%%%%
\makeindex

%%%%% Useful macros %%%%%
\newcommand{\latinloc}[1]{\ifx\undefined\lncs\relax\emph{#1}\else\textrm{#1}\fi\xspace}
\newcommand{\etc}{\latinloc{etc}}
\newcommand{\eg}{\latinloc{e.g.}}
\newcommand{\ie}{\latinloc{i.e.}}
\newcommand{\cad}{c'est-à-dire }
\newcommand{\st}{\ensuremath{\text{\xspace s.t.\xspace}}}

%%%% Definition des couleur %%%%

\newcommand\couleurb[1]{\textcolor{SteelBlue}{#1}}
\newcommand\couleurr[1]{\textcolor{DarkRed}{#1}}


%% number page style style %%%%%%%%%%%%%%%%%%%%%%%%%%%%%%%%%%%%%%%%%%%%%%%%%%%%%%

\pagestyle{plain}
%\pagestyle{empty}
%\pagestyle{headings}
%\pagestyle{myheadings}



%% chapters style %%%%%%%%%%%%%%%%%%%%%%%%%%%%%%%%%%%%%%%%%%%%%%%%%%%%%%
%% You may try several styles (see more in the memoir manual).

%\chapterstyle{veelo}
%\chapterstyle{chappell}
%\chapterstyle{ell}
%\chapterstyle{ger}
%\chapterstyle{pedersen}
%\chapterstyle{verville}
\chapterstyle{madsen}
%\chapterstyle{thatcher}


%%%%% Report Title %%%%%
\title{Sniff Hynesim}
\author{\textsc{Rigaud Michaël}}
\date{\today}
\doctype{Status Report}
\promo{promo 2017}
\etablissement{\textsc{Ensta} Bretagne\\2, rue François Verny\\
  29806 \textsc{Brest} cedex\\\textsc{France}\\Tel +33 (0)2 98 34 88 00\\ \url{www.ensta-bretagne.fr}}
\logoEcole{\includegraphics[height=4.2cm]{logo_ENSTA_Bretagne_Vertical_CMJN}}



%%%%%%%%%%%%%%%%%% DEBUT DU DOCUMENT
\begin{document}

\maketitle
\thispagestyle{empty}
\newpage

\tableofcontents


%%%%%%%%%%%%%%%%% INTRODUCTION

\chapter*{Introduction}
\addcontentsline{toc}{chapter}{Introduction}

The cyber security is one of the major thread of the 21th century, and attackers use techniques more and more
sophisticated. So one of the most important aim for cyber security engineer is to find a way to detect and stop
attacks. To do that effectively cyber engineer need to analyze cyber attack to find a way to detect them. A
solution is use simulators of network and information system to reproduce as much as they want, without injury, and
huge agility scenario of cyber attack. To do that, the ENSTA Bretagne has decided as many company like Thales or
the DGA to use Hynesim\footnote{This software is presented in the chapter \ref{chap:hynesim}}.

The aim of this project is elaborate a solution to alert when a strategy of attack is spotted out. To do that, we
have to create pattern of attack and use an IDS\footnote{Intrusion detection system} to alert us. To verify the
solution and create pattern as exhaustive as possible, we will use Hynesim.

To begin, we will present Hynesim and the advantages of this software. Then, we are going to present the aim of an
IDS and the most popular IDS. And to finish we will present the aim of this project and the organization of the
project.

\newpage
%%%%%%%%%%%%%%%%%%%%%%%%


%%%%%%%%%%%%%%%%%%%%%%%%%%%%%%%%%%%%%%%%%%%%%%%%%%%%%%%%%%%%%%%%%%%%%%%%%%%%%%%%%%%%%%%%%%%%%%%%%%%%%%%%%%%%%%%%%%%
% Construction du rapport:
%
% Partie 1 : Etude Biblio et organisation (cf status report)
% Ajout d'un chapitre sur les SIEM
% Partie 2 : Installation
% | Chapitre 2.1 : Installation de Hynesim et son utilisation
% | Chapitre 2.2 : Installation de Selks et son utilisation
% | Chapitre 2.3 : Installation de Prelude et son utilisation
% Partie 3 : Expérience
% | Chapitre 3.1 : Attaques menés
% | Chapitre 3.2 : Résultats
% - Chapitre 3.3 : amélioration
%
%%%%%%%%%%%%%%%%%%%%%%%%%%%%%%%%%%%%%%%%%%%%%%%%%%%%%%%%%%%%%%%%%%%%%%%%%%%%%%%%%%%%%%%%%%%%%%%%%%%%%%%%%%%%%%%%%%%


\part{Status report}
  
\chapter{Hynesim}
\label{chap:hynesim}

Firstly, we will present Hynesim. In fact, we will use this tool to create all our network and test our solution so
it is important to introduce it.

\section{Presentation}

\begin{figure}[h]
  \centering
  \includegraphics[width=0.5\textwidth]{hynesim}
  \caption{Hynesim logo}
  \label{fig:hynesim}
\end{figure}


\definition{Hynesim}{Means HYbrid NEtwork SIMulation, is a distribute platform of simulation of information system
  developed by Diateam. \cite{hynesim}}

The platform was initially developed by Diateam for DGA MI (Maitrise de l'information) to create virtual network.
But now is a major project to develop information system and automatize cyber security attacks. This project has
two version, an open source version and an professional version. The open source version as less option, but we
will use this version for this project.

\section{Architecture}

To work, Hynesim need a server with on it the main software. This software is the virtualization part. It manage
virtual machine and network.

Moreover, to see virtual machine and interact with them, users need to have the client interface. This interface
can be install on a simple computer.

To add virtual machine on Hynesim, we need to create them on Virtual Box\footnote{More information here:
  \url{https://www.virtualbox.org/}} or VMWare\footnote{More information here: \url{https://www.vmware.com/}} and
then import them on Hynesim.

\begin{figure}[h]
  \centering
  \includegraphics[width=\textwidth]{hynesim_network}
  \caption{An example of an client Hynesim interface}
\end{figure}





%%% Local Variables:
%%% mode: latex
%%% TeX-master: "../rapport_de_base"
%%% End:

  \chapter{IDS}

In this subject, we will have to create probe to detect attack. It is the aim of IDS that we will present.

\section{Presentation}

\definition{IDS}{An intrusion detection system (IDS) inspects all inbound and outbound network activity and
identifies suspicious patterns that may indicate a network or system attack from someone attempting to break into
or compromise a system. \cite{webopedia}}



There is many type of IDS:
\begin{itemize}
\item NIDS, network IDS. They listen and analyze the network and detect attack from network packet. They are the
  most interesting for our subject, so this document will focus primarily on this type of IDS.
\item HIDS, host IDS. These IDS are on a system and they detect intrusion inside it.
\item Hybrid IDS. They are composed with NIDS and HIDS.
\item IPS. They are NIDS with active functions which permit to stop attackers.
\item KIDS/KIPS, kernel IDS. They are type of HIDS. They are on the system kernel. They are more effective and
slower than HIDS.
\end{itemize}

In the following of this document we will talk about NIDS. But to detect attack there is also many methods.

%%%%%%%%%%%%%%%%%%%%%%%%%%%%%%%%%%%%%%%%%%%%%%%%%%%%%%%%%%%%%%%%%%%%%%%%%%%%%%
% TODO: mettre en avant les but d un ids. ie rapidite et pas de fausse alarm
%%%%%%%%%%%%%%%%%%%%%%%%%%%%%%%%%%%%%%%%%%%%%%%%%%%%%%%%%%%%%%%%%%%%%%%%%%%%%%

\section{Detection methods}

\subsection{Misuse detection}

This technique is the simplest. It use attack signature to raise alert. In fact, all attacks have a particularity,
if we detect this particularity we can detect the attackers. There is three sub methods.

\subsubsection{Pattern matching}

In this technique we have a based of signature and the IDS is looking for the pattern. If the pattern match
perfectly, this IDS raise an alert.

The problem is, only attacks which are in our based can be detected. So if there is a new attack (zero day), or if
the attack is not perfectly the same, we can't detect it.

However, this method is much used because it is high-performance, and with this method the IDS don't raise a lot of
false alerts

\subsubsection{Dynamic pattern matching}

In this techniques the IDS is also based on signature but this data base is dynamic. In other words, the IDS has
the faculty of adaptation and learning. The IDS improve his data base of signature automatically.

\subsubsection{Protocol analysis}

The last sub method we will present is the protocol analysis. This technique is based on the verification of
protocol. The IDS will check if flows are compliant with RFC\footnote{Requests for Comments, is a type of
  publication from the Internet Engineering Task Force (IETF) and the Internet Society (ISOC), the principal
  technical development and standards-setting bodies for the Internet.} standards. It will verify parameters of
packets and fields of them. An IDS can check many protocol as FTP, HTTP, ICMP, \dots

The advantages of this methods is that we can detect unknown attacks in contrary of pattern matching. However,
software publishers don't often respect RFC so we this technique is not always very effective.





\subsection{Anomaly detection}

This technique consists in detecting an intrusion with the analysis of the user's past behavior. So the IDS should
create a profile of users from his use and raise alert when there is an event outside this profile. To create
profile the IDS could use machine learning.
~\\

\begin{tabular}{|p{0.45\textwidth}|p{0.45\textwidth}|} \hline
Advantages                                                                                                 & Drawbacks \\ \hline The IDS should be able to detect every type of
attack even unknown (zero days) attacks.                                                                   & This method is not reliable. Every
alteration of the use create an alert                                                                                  \\
  \hline The IDS is autonomous                                                                             & This method
need a learning period. In fact, this method need to learn the habit of users,
so we need a period without attacks                                                                                    \\ \hline & An hackers can need only time. In fact,
if the attacker arrive to create a new profile after month with his attacks, he
could attack silently                                                                                                  \\ \hline

\end{tabular}


\subsubsection{Probabilistic method}
Bayésien network is a learning machine based on probability. The IDS will create a probabilistic model and if the
user and will raise an alert if the user don't respect this model.

For example, we know that in 90\% of cases in an HTTP request the first parameter is GET after a connection to the
port 80.


%
% \subsubsection{Statistic method}
% \subsubsection{Other method}

% It is possible to use every type of machine learning. In particularly, neural
% network.


\section{Main IDS}

There is many IDS on the market. The most popular open source solution are Snort, Suricata, Bro, Fail2Ban, ACARM
... There is also some tools <<all in one>>. It is usually OS\footnote{Operating system} with an IDS, a tool to
analyze alert, a tool to create rules,... The most popular are SELKS and ELKS. A description of SELKS is available
page \pageref{chap:selks}.




%%% Local Variables:
%%% mode: latex
%%% TeX-master: "../rapport_de_base"
%%% End:

  
\chapter{SIEM}
\label{chap:project}

After some new discussion with my tutor, we decide to had a new component in our architecture. So we will add a
SIEM to analyze all alert and take care of server's log. So we will add in this status report a part about SIEM
which was not initially.

\section{Description}

\definition{SIEM}{In the field of computer security, security information and event management (SIEM) software
  products and services combine security information management (SIM) and security event management (SEM). They
  provide real-time analysis of security alerts generated by network hardware and applications.}


%%%%%%%%%%%%%%%%%%%%%%%%%%%%%%%%%%%%%%%%%%%%%%%%%%%%%%%%%%%%%%%%%%%%%%%%%%%%%%%%%%%%%%%%%%%%%%%%%%%%%%%%%%%%%%%%%%%
% TODO: Completer cette étude biblio
%%%%%%%%%%%%%%%%%%%%%%%%%%%%%%%%%%%%%%%%%%%%%%%%%%%%%%%%%%%%%%%%%%%%%%%%%%%%%%%%%%%%%%%%%%%%%%%%%%%%%%%%%%%%%%%%%%%




%%% Local Variables:
%%% mode: latex
%%% TeX-master: "../rapport_de_base"
%%% End:

  
\chapter{Position of the Project}
\label{chap:project}


\section{Resume}

This project is to automate a sniffer to detect attack. The most important fact is that our sensors need to detect
a strategy of attack and not only patterns of attack. After our bibliographic studies, it is simple to understand
that we will use an IDS with an anomaly detection method. We will use SELKS\footnote{A description of SELKS is
  available page \pageref{chap:selks}} because it is the main free open source
IDS available on the market. A description of our technical choice is explain on the next section.

As it is suggest in the description of the project deliver by our professor, we also use Hynesim to automate test
and simulate our solution. In this way, we will have the ability to justify the effectiveness of our solution.
To do that we will run many possible attack scenarios.

\section{Technical choice}

We will firstly realize a network infrastructure on Hynesim. We will do a basic infrastructure with only one server.
Then, we will put on our network a SLEKS server as a simple IDS and we will configure. We choose, to implement firstly
a pattern matching method. We will also put an attacker on the network and we will realize simple attack to test our
infrastructure. It is possible to see the infrastructure on the figure \ref{fig:network_hynesim}.

\begin{figure}[h]
  \centering
  \includegraphics[width=0.7\textwidth]{reseau_hynesim}
  \caption{Hynesim network architecture}
  \label{fig:network_hynesim}
\end{figure}

After this installation, we will try to improve the system. we will implement an anomaly detection method. In this
way, the IDS will have the ability to detect a will to attack. One of the most important difficulty will be to not
raise alert for a normal utilization.

\section{Forecasting organization}

\subsection{Kanban}

To realize this project, we decided to use some tools to arrange our work. First of all, we decided to use a agile
technique of management which name Kanban.~\\

\definition{Kanban}{Kanban is a new technique for managing a software development process in a highly efficient
  way. Kanban underpins Toyota's "just-in-time" (JIT) production system. kanban system consists of a big board on
  the wall with cards or sticky notes placed in columns with numbers at the top \cite{peterson:kanban}}


Kanban is an inventory-control system to control the supply chain. It use a board with columns. Each columns
represent a status, for example: to do, doing, done. In each column we put <<notes>> which represent a task.
Moreover, each column have a maximum number of notes authorized.


\begin{figure}[h]
  \centering
  \includegraphics[width=\textwidth]{kandboard}
  \caption{Kanboard of this project}
  \label{fig:kanboard}
\end{figure}


Limiting the amount of task, at each step in the process, prevents overproduction and revels bottlenecks
dynamically. In fact, with this technique it is possible to have a better overview of the project and control it
dynamically.
~\\

For this project, we use Kanboard\cite{guillot:kanboard} self-hosted on our own Yunohost server.\footnote{ It is possible to
  see our kanboard at this link:
  \url{https://mic-rigaud.fr/kanboard/?controller=BoardViewController&action=readonly&token=10ea65eca908023dbcd8bc8dce75791c7a14d67912627dafaa5b71033222}}.
You can see at the figure \ref{fig:kanboard} the kanboard of this project. Each color represent a category of task:
blue: improvement , purple: installation, red : experience, green: status report, and gray: report.

With this tool, it is also possible to see tasks as a Grantt diagram. The figure \ref{fig:grantt} present our
Grantt diagram.

\begin{figure}[h]
  \centering
  \includegraphics[width=\textwidth]{grant}
  \caption{Grantt diagram of the project}
  \label{fig:grantt}
\end{figure}


\subsection{Github}

To realize this project, we also decided to use Git and Github as Git server. ~\\

\definition{Git}{Git is a version control system for tracking changes in computer files and coordinating work on
  those files among multiple people.}

Git permit to control version of our work and have a real showcase of it for our tutor. The figure
\ref{fig:github} is a screenshot of our Github server.

\begin{figure}[h]
  \centering
  \includegraphics[width=\textwidth]{github}
  \caption{Github server}
  \label{fig:github}
\end{figure}




%%% Local Variables:
%%% mode: latex
%%% TeX-master: "../rapport_de_base"
%%% End:


\part{Technical study}
  \input{Partie/Install_hynesim}
  
\chapter{SELKS}
\label{chap:selks}



%%% Local Variables:
%%% mode: latex
%%% TeX-master: "../rapport_de_base"
%%% End:

  
\chapter{Prelude}
\label{chap:prelude}


%%% Local Variables:
%%% mode: latex
%%% TeX-master: "../rapport_de_base"
%%% End:


\part{Improvement}
  \input{Partie/Attacks}


%%%% CONCLUSION %%%%%%%%%

\chapter*{Conclusion}
\addcontentsline{toc}{chapter}{Conclusion}


After this bibliographic study, to implement our network sniffer which detect strategy of attack we have to improve
an IDS. In fact, IDS have the ability to detect attack with many methods. Here, we have to use an anomaly detection
method to find attackers. However, as we see, this method has many disadvantages so we have to improve it.

Moreover, as it was advise by the subject we will use Hynesim. In fact, Hynesim is a very interesting tool to
virtualize and simulate network. By this way, we could test under many attacks our solution to secure network.

After this study, we have only a partial view of how implement this detection method. So one of the most important
work for the next two weeks will be find the way to detect strategy of attacks. It is a difficult objective but
also one of the most interesting for the security of our networks.


\newpage

%%%% ANNEXE %%%%%%%%%%%%

\part*{Annexe}
\appendix
\nocite{*}
%
\chapter{SELKS}
\label{chap:selks}



%%% Local Variables:
%%% mode: latex
%%% TeX-master: "../rapport_de_base"
%%% End:

\newpage
 \listoffigures
 \printindex
 \bibliographystyle{plain}
  \bibliography{biblio}

\end{document}
%%%%%%%%%%%%%%%%% FIN DU DOCUMENT
%%% Local Variables:
%%% mode: latex
%%% TeX-master: t
%%% End:
